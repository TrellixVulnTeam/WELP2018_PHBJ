\documentclass[12pt]{article}
\usepackage[spanish]{babel}
\selectlanguage{spanish}
\usepackage[utf8]{inputenc}
\usepackage{graphicx}
%\usepackage[margin=1in]{geometry}

\setlength{\parskip}{1em}
\setlength{\parindent}{0em}

\newcommand{\bi}{\begin{itemize}}
\newcommand{\ei}{\end{itemize}}

\newcommand{\cue}{\mbox{}
  \fbox{\sf Next slide}
}

\newcommand{\fig}[1]{
  \newpage  
  \begin{center}
    \includegraphics[width=3in]{../singlepages/pg_00#1.pdf}
  \end{center}
  
}

\begin{document}
\Large


\fig{01}
\bi
\item Buenas tardes y gracias por venir.
\item Soy Geoffrey Matthews.
\item Soy  profesor de informática.
\item He estado aquí en Western por más de 30 años.
\item En los últimos años, tres cosas cambiaron, y eso cambió mi vida.
  \ei
  
\fig{03}

\bi
\item Primero, he aprendido español.
\item \cue
\item En 2012, decidimos (mi esposa y
yo) tomar una vacacion a España, así que decidí aprender un poco de
español.
\ei

\fig{04}

\bi
\item El programa WELP en Western estaba ofreciendo cursos de
  español gratis!
\item ¿Qué tenía que perder?
\item  Así que empecé.
\ei

\fig{05}

\bi
\item He descubierto que la mejor forma de aprender idiomas es con clases
  pequeñas o, mejor aún, privadas.
\item Así que empecé muy pronto a hablar
con los estudiantes de WELP, como Diane, Eldred, Giselle, Majo,
Gloria, y otros.
\item Ellos vinieron a mi oficina y charlar conmigo por un
  rato cada semana.
\item ¡Es fantastico!
\item ¡Muchísimas gracias a todos!
\ei

\fig{06}
\bi
\item Mis visitas a España fue unos éxitos fantásticos.
\item  Me encanta viajar,
conocer gente, y la experiencia de chatear en un idioma extranjero.
\ei

\fig{07}

\bi
\item Viví con una familia en México por una semana de inmersión intensiva
en español, y me encantó la experiencia.
\item
Aqui tenemos yo, mi esposa, mi ``madre mexicana,'' y mi sobrina
honorifica.  
\item
  Me encanta el español.
\item Espero que parezca que la resulta fue un gran
éxito, ¿no?  ¿Que piensas de mi español?
\ei

\fig{09}

\bi
\item Eso fue lo primero que pasó.
La segunda cosa que pasó también comenzó
  alrededor del año 2012.
\item \cue
\item
  Aunque la informática ha sido una opción
de carrera viable durante varias décadas, en los últimos años ha
estallado el número de empleos disponibles.
\item Y eso no muestra signos
  de desaceleración.
\item La Bureau of Labor Statistics de los Estados
Unidos predicen que hasta el 2022 más del 75\% de todos los empleos
nuevos en las ciencias, matemáticas, y engeniería, serán en el campo
de la informática.
\item
  Sí, habrá más empleos para científicos
informáticos que para todos los demás científicos {\em combinado.}

\item ¡Más de tres de cada cuatro!

\ei
\fig{10}

\bi
\item
  Además, estos trabajos son algunos de los mejores y más remunerados.
  en América.
\item Cuando la CBS publicaron su reciente lista de los
``Mejores 11 Empleos en América para 2017'', cinco de los diez
primeros requieren una licenciatura en informática. 
\item
 Los salarios que
allí figuran, alrededor de 100.000 dólares al año, son típicos de lo
que nuestros estudiantes están recibiendo.
\item
Y esos no son los salarios promedios.  ¡Esos son sueldos iniciales!
\ei

\fig{11}

\bi
\item
Además, los empleos son tan interesante.
\item
Aqui tenemos un gráfico de
un trabajo de investigación por Saiph Savage, una profesora de
informatica.
\item
Se trata de las milicias y otros ciudadanos durante un
crisis.
\item
  Ella utilizó los datos de las redes sociales para entender
mejor.
\ei

\fig{12}

\bi
\item
A pesar de que hay tantos buenos empleos en este campo,
no hay suficiente empleados calificados.
\item
Se necesita una licenciatura en informática, y no hay muchos.
\item
  Un informe de la Alianza Tecnológica dice que llenaremos sólo el 27\% del total de las vacantes de empleo STEM en nuestro estado hasta el 2025 si continuamos graduándonos al mismo ritmo que lo estamos haciendo hoy en día.
\ei

\fig{14}
\bi
\item
Eso fue lo segundo que pasó.
\item
Primero, aprendí Español.
\item
Segundo, el crisis de informaticá.
\item
Hay muchos empleos en informática, pero poco empleados.
\item \cue
\item
  La tercera cosa que ocurrió en los últimos años fue la explosión en el
  población latina en los estados unidos.
\item
 Especialmente en nuestra
 zona.
\item
  En el condado de Whatcom, la población latina actual está cerca
del 10\%.  En el condado de Skagit, cerca de 20\%. 
\ei

%\fig{15}
\fig{16}

\bi
\item
Entonces, tenemos muchos empleos, pocos empleados, y muchos latinos.
¿Que podemos hacer?
\item
¿Cual es la solución de la crisis en informática?
\item \cue
Más minorías y mujeres!
\item
Aqui podemos ver que los blancos y los asiáticos están
sobrerepresentadas en este campo, pero otras minorías y las mujeres
están muy poco representadas.
\item
  La barra de azul oscuro es la
representación en informática, o en ``alta tecnologia,'' y la barra de
azul claro es la representación en todos campos.
\item
Una representación justa debe duplicar el número de latinos en el
campo.
\item
Eso ayudaría a aliviar la crisis.
\ei

\fig{17}

\bi
\item El negocio entiende este problema.
\item
  Hay muchas becas y pasantías disponible
  para las minorías.
\item
  El Washington State Opportunity Scholarship es
posible gracias a una asociación público-privada, con el apoyo
significativo de los socios fundadores Boeing y Microsoft y dólares de
contrapartida del Estado de Washington.
\ei


\fig{18}

\bi
\item
  ¡Y eso era sólo una de ellas!
\item
  ¡Hay mucho más!
\item
  ¡Dinero gratis!
  \ei
  
\fig{20}

\bi
\item
Entonces, ¿qué podemos hacer?
\item
Y recuerda, hablo español.
\item \cue
\item Por supuesto.  Me propongo ir a las escuelas y explicar estos hechos. 
\item
Me gustaría anunciar que he recibido un premio de diversidad y
justicia social de Western, dándome tiempo libre de enseñar el proximo
año. 
\ei

\fig{21}
\bi
\item Cuando voy a las escuelas, también voy a mostrarles que la
  informática es divertida.
\item
Ahorita, me gustaría darles una demostración de informática.
\item
Voy a crear un videojuego in diez minutos.
\item
Es un ejemplo del tipo de cosas que quiero hacer para inspirar a los
estudiantes a estudiar informática.
\ei

\fig{22}
\bi
\item
Finalmente, quiero hablar un poco sobre mi historia familiar.  Mi
abuelo era de Cornwall.
\item
  La gente de Cornwall fue perseguida por los
  ingleses.
\item Los niños en la escuela fueron golpeados por hablar su
  propio idioma en lugar del inglés.
\item
  Dejó Inglaterra y llegó a América
  para trabajar como herrador de caballos en Filadelfia.
\item
  Su nieto, yo,
  ahorita soy miembro de la clase media.
\item
  No sufro persecución.
\item Tengo
  un buen trabajo, familia, casa, todo.
\item
  Las cosas pueden cambiar más
rápido de lo que puedan imaginar.
\ei

\fig{23}




\end{document}

